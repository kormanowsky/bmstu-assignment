\chapter*{ОПРЕДЕЛЕНИЯ, ОБОЗНАЧЕНИЯ И СОКРАЩЕНИЯ}
\addcontentsline{toc}{chapter}{ОПРЕДЕЛЕНИЯ, ОБОЗНАЧЕНИЯ И СОКРАЩЕНИЯ}
В настоящей расчетно-пояснительной записке применяют следующие термины с соответствующими определениями.

\begin{longtable}{|p{.4\textwidth - 2\tabcolsep}|p{.6\textwidth - 2\tabcolsep}|}
    \caption{Термины и определения}\label{tbl:terms} \\\hline
    Термин                                      & Определение                                                     \\
    \hline
    \endfirsthead
    \caption*{Продолжение таблицы~\thetable }\\
    \hline
    Термин                                      & Определение                                                     \\
    \hline
    \endhead
    \endfoot
    Взвешенный ориентированный граф & Пара
    \begin{equation*}
        W = (G, \omega),
    \end{equation*} где $G = (V, E)$ --- ориентированный граф,
    $\omega(e)$ --- весовая функция (функция разметки),
    ставящая в соответствие каждой дуге $e \in E$ некоторое значение, например, время
    прохождения по дуге \\\hline

\end{longtable}

В настоящей расчетно-пояснительной записке применяют следующие сокращения и обозначения.
\begin{description}[leftmargin=0pt]
    \item  $A$ --- первая буква кириллического алфавита.
\end{description}